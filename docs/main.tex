\documentclass[]{article}

\usepackage{amsmath}
\usepackage{syntax}

\title{Grammar Definition}
\author{Dominik Drexler}
\date{\today}

\begin{document}
\maketitle

% https://tex.stackexchange.com/questions/24886/which-package-can-be-used-to-write-bnf-grammars
\begin{grammar}
    \newcommand{\plus}{\textsuperscript{+}}
    \newcommand{\typing}{\textsuperscript{:typing~}}
    \newcommand{\fluents}{\textsuperscript{:fluents~}}

    <domain> ::= ( define ( domain <name> )\newline
    [<require-def>] \newline
    [<types-def>] \newline
    [<constants-def>] \newline
    [<predicates-def>] )

    <require-def> ::= :strip | :typing

    <types-def> ::= ( :types <typed list(name)> )

    <constants-def> ::= ( :constants <typed list(name)> )

    <predicates-def> ::= ( :predicates <atomic formula skeleton>* )

    <atomic formula skeleton> ::= ( <name> <typed list (variable)> )

    <typed list (x)> ::= x*

    <typed list (x)> ::= x\plus - <type> <typed list (x)>

    <type> ::= <name> | ( either <type>\plus )

    <type> ::= ( <type> )

    <name> ::= <letter> <any char>*

    <variable> ::= ?<name>

    <any char> ::= <letter> | <digit> | - | _

    <letter> ::= a..z | A..Z

    <digit> ::= 0..9

\end{grammar}

\nocite{mcdermott-et-al-1998}

\bibliographystyle{plain}
\bibliography{refs}

\end{document}